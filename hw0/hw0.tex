\documentclass[12pt, fullpage,letterpaper]{article}

\usepackage[margin=1in]{geometry}
\usepackage{url}
\usepackage{amsmath}
\usepackage{amssymb}
\usepackage{xspace}
\usepackage{graphicx}

\newcommand{\semester}{Spring 2019}
\newcommand{\assignmentId}{0}
\newcommand{\releaseDate}{7 January, 2019}
\newcommand{\dueDate}{11:59pm, 16 January, 2019}

\newcommand{\bx}{{\bf x}}
\newcommand{\bw}{{\bf w}}

\title{CS 5350/6350: Machine Learining \semester}
\author{Homework \assignmentId}
\date{Handed out: \releaseDate\\
  Due: \dueDate}

\begin{document}
\maketitle

% Math commands by Thomas Minka
\newcommand{\var}{{\rm var}}
\newcommand{\Tr}{^{\rm T}}
\newcommand{\vtrans}[2]{{#1}^{(#2)}}
\newcommand{\kron}{\otimes}
\newcommand{\schur}[2]{({#1} | {#2})}
\newcommand{\schurdet}[2]{\left| ({#1} | {#2}) \right|}
\newcommand{\had}{\circ}
\newcommand{\diag}{{\rm diag}}
\newcommand{\invdiag}{\diag^{-1}}
\newcommand{\rank}{{\rm rank}}
% careful: ``null'' is already a latex command
\newcommand{\nullsp}{{\rm null}}
\newcommand{\tr}{{\rm tr}}
\renewcommand{\vec}{{\rm vec}}
\newcommand{\vech}{{\rm vech}}
\renewcommand{\det}[1]{\left| #1 \right|}
\newcommand{\pdet}[1]{\left| #1 \right|_{+}}
\newcommand{\pinv}[1]{#1^{+}}
\newcommand{\erf}{{\rm erf}}
\newcommand{\hypergeom}[2]{{}_{#1}F_{#2}}

% boldface characters
\renewcommand{\a}{{\bf a}}
\renewcommand{\b}{{\bf b}}
\renewcommand{\c}{{\bf c}}
\renewcommand{\d}{{\rm d}}  % for derivatives
\newcommand{\e}{{\bf e}}
\newcommand{\f}{{\bf f}}
\newcommand{\g}{{\bf g}}
\newcommand{\h}{{\bf h}}
%\newcommand{\k}{{\bf k}}
% in Latex2e this must be renewcommand
\renewcommand{\k}{{\bf k}}
\newcommand{\m}{{\bf m}}
\newcommand{\mb}{{\bf m}}
\newcommand{\n}{{\bf n}}
\renewcommand{\o}{{\bf o}}
\newcommand{\p}{{\bf p}}
\newcommand{\q}{{\bf q}}
\renewcommand{\r}{{\bf r}}
\newcommand{\s}{{\bf s}}
\renewcommand{\t}{{\bf t}}
\renewcommand{\u}{{\bf u}}
\renewcommand{\v}{{\bf v}}
\newcommand{\w}{{\bf w}}
\newcommand{\x}{{\bf x}}
\newcommand{\y}{{\bf y}}
\newcommand{\z}{{\bf z}}
%s\newcommand{\l}{\boldsymbol{l}}
\newcommand{\A}{{\bf A}}
\newcommand{\B}{{\bf B}}
\newcommand{\C}{{\bf C}}
\newcommand{\D}{{\bf D}}
\newcommand{\E}{{\bf E}}
\newcommand{\F}{{\bf F}}
\newcommand{\G}{{\bf G}}
\renewcommand{\H}{{\bf H}}
\newcommand{\I}{{\bf I}}
\newcommand{\J}{{\bf J}}
\newcommand{\K}{{\bf K}}
\renewcommand{\L}{{\bf L}}
\newcommand{\M}{{\bf M}}
\newcommand{\N}{\mathcal{N}}  % for normal density
%\newcommand{\N}{{\bf N}}
\renewcommand{\O}{{\bf O}}
\renewcommand{\P}{{\bf P}}
\newcommand{\Q}{{\bf Q}}
\newcommand{\R}{{\bf R}}
\renewcommand{\S}{{\bf S}}
\newcommand{\T}{{\bf T}}
\newcommand{\U}{{\bf U}}
\newcommand{\V}{{\bf V}}
\newcommand{\W}{{\bf W}}
\newcommand{\X}{{\bf X}}
\newcommand{\Y}{{\bf Y}}
\newcommand{\Z}{{\bf Z}}

% this is for latex 2.09
% unfortunately, the result is slanted - use Latex2e instead
%\newcommand{\bfLambda}{\mbox{\boldmath$\Lambda$}}
% this is for Latex2e
\newcommand{\bfLambda}{\boldsymbol{\Lambda}}

% Yuan Qi's boldsymbol
\newcommand{\bsigma}{\boldsymbol{\sigma}}
\newcommand{\balpha}{\boldsymbol{\alpha}}
\newcommand{\bpsi}{\boldsymbol{\psi}}
\newcommand{\bphi}{\boldsymbol{\phi}}
\newcommand{\boldeta}{\boldsymbol{\eta}}
\newcommand{\Beta}{\boldsymbol{\eta}}
\newcommand{\btau}{\boldsymbol{\tau}}
\newcommand{\bvarphi}{\boldsymbol{\varphi}}
\newcommand{\bzeta}{\boldsymbol{\zeta}}

\newcommand{\blambda}{\boldsymbol{\lambda}}
\newcommand{\bLambda}{\mathbf{\Lambda}}
\newcommand{\bOmega}{\mathbf{\Omega}}
\newcommand{\bomega}{\mathbf{\omega}}
\newcommand{\bPi}{\mathbf{\Pi}}

\newcommand{\btheta}{\boldsymbol{\theta}}
\newcommand{\bpi}{\boldsymbol{\pi}}
\newcommand{\bxi}{\boldsymbol{\xi}}
\newcommand{\bSigma}{\boldsymbol{\Sigma}}

\newcommand{\bgamma}{\boldsymbol{\gamma}}
\newcommand{\bGamma}{\mathbf{\Gamma}}

\newcommand{\bmu}{\boldsymbol{\mu}}
\newcommand{\1}{{\bf 1}}
\newcommand{\0}{{\bf 0}}

% \newcommand{\comment}[1]{}

\newcommand{\bs}{\backslash}
\newcommand{\ben}{\begin{enumerate}}
\newcommand{\een}{\end{enumerate}}

 \newcommand{\notS}{{\backslash S}}
 \newcommand{\nots}{{\backslash s}}
 \newcommand{\noti}{{\backslash i}}
 \newcommand{\notj}{{\backslash j}}
 \newcommand{\nott}{\backslash t}
 \newcommand{\notone}{{\backslash 1}}
 \newcommand{\nottp}{\backslash t+1}
% \newcommand{\notz}{\backslash z}

\newcommand{\notk}{{^{\backslash k}}}
%\newcommand{\noti}{{^{\backslash i}}}
\newcommand{\notij}{{^{\backslash i,j}}}
\newcommand{\notg}{{^{\backslash g}}}
\newcommand{\wnoti}{{_{\w}^{\backslash i}}}
\newcommand{\wnotg}{{_{\w}^{\backslash g}}}
\newcommand{\vnotij}{{_{\v}^{\backslash i,j}}}
\newcommand{\vnotg}{{_{\v}^{\backslash g}}}
\newcommand{\half}{\frac{1}{2}}
\newcommand{\msgb}{m_{t \leftarrow t+1}}
\newcommand{\msgf}{m_{t \rightarrow t+1}}
\newcommand{\msgfp}{m_{t-1 \rightarrow t}}

\newcommand{\proj}[1]{{\rm proj}\negmedspace\left[#1\right]}
\newcommand{\argmin}{\operatornamewithlimits{argmin}}
\newcommand{\argmax}{\operatornamewithlimits{argmax}}

\newcommand{\dif}{\mathrm{d}}
\newcommand{\abs}[1]{\lvert#1\rvert}
\newcommand{\norm}[1]{\lVert#1\rVert}

%miscellaneous symbols
\newcommand{\ie}{{{i.e.,}}\xspace}
\newcommand{\eg}{{{\em e.g.,}}\xspace}
\newcommand{\EE}{\mathbb{E}}
\newcommand{\VV}{\mathbb{V}}
\newcommand{\sbr}[1]{\left[#1\right]}
\newcommand{\rbr}[1]{\left(#1\right)}
\newcommand{\cmt}[1]{}


\input{general-instructions}


\section*{Basic Knowledge Review}
\label{sec:q1}

%problem 1, decide wether dependency & indendency
%2, prove p(A + B) <= P(A) + P(B)
%3, prove P(\sum_i A_i) \le \sum_i p(A_i)
%4. given two a joint Gaussian, calculate the conditional Guassian distribution
%5. prove E(X) = E(E(X|Y))
%4, prove V(E(X)) = E(V(X))
%5. prove V(Y) = EV(Y|X) + VE(Y|X)

%independency, conditional distribution, expectation, variance, basic properties
%gradient calcualtion, logistic function, second derivatives
%
\begin{enumerate}
\item~[5 points] We use sets to represent events. For example, toss a fair coin $10$ times, and the event can be represented by the set of ``Heads" or ``Tails" after each tossing. Let a specific event $A$ be ``at least one head". Calculate the probability that event $A$ happens, i.e., $p(A)$.
\item~[10 points] Given two events $A$ and $B$, prove that 
\[
p(A \cup B) \le p(A) + p(B).
\]
When does the equality hold?
\item~[10 points] Let $\{A_1, \ldots, A_n\}$ be a collection of events. Show that
\[
p(\cup_{i=1}^n A_i) \le \sum_{i=1}^n p(A_i).
\]
When does the equality hold? (Hint: induction)
%\item~[5 points] Given three events $A$, $B$ and $C$, show that
%\[
%p(A\cap B\cap C) = p(A|B\cap C)p(B|C)p(C)
%\]
\item~[20 points]  We use $\EE(\cdot)$ and $\VV(\cdot)$ to denote a random variable's mean (or expectation) and variance, respectively. Given two discrete random variables $X$ and $Y$, where $X \in \{0, 1\}$ and $Y \in \{0,1\}$. The joint probability $p(X,Y)$ is given in as follows:
\begin{table}[h]
        \centering
        \begin{tabular}{ccc}
        \hline\hline
         & $Y=0$ & $Y=1$ \\ \hline
         $X=0$ & $1/10$ & $2/10$ \\ \hline
         $X=1$  & $3/10$ & $4/10$ \\ \hline\hline
        \end{tabular}
        %\caption{Training data for the alien invasion problem.}\label{tb-alien-train}
        \end{table}
	
        \begin{enumerate}
            \item~[10 points] Calculate the following distributions and statistics. 
            \begin{enumerate}
            \item the the marginal distributions $p(X)$ and $p(Y)$
            \item the conditional distributions $p(X|Y)$ and $p(Y|X)$
            \item $\EE(X)$, $\EE(Y)$, $\VV(X)$, $\VV(Y)$
            \item  $\EE(Y|X=0)$, $\EE(Y|X=1)$,  $\VV(Y|X=0)$, $\VV(Y|X=1)$ 
            \item  the covariance between $X$ and $Y$
            \end{enumerate}
            \item~[5 points] Are $X$ and $Y$ independent? Why?
            \item~[5 points] When $X$ is not assigned a specific value, are $\EE(Y|X)$ and $\VV(Y|X)$ still constant? Why?
        \end{enumerate}
\item~[10 points] Assume a random variable $X$ follows a standard normal distribution, \ie $X \sim \N(X|0, 1)$. Let $Y = e^X$. Calculate the mean and variance of $Y$.
\begin{enumerate}
	\item $\EE(Y)$
	\item $\VV(Y)$
\end{enumerate}

\item~[20 points]  Given two random variables $X$ and $Y$, show that 
\begin{enumerate}
\item $\EE(\EE(Y|X)) = \EE(Y)$
\item
$\VV(Y) = \EE(\VV(Y|X)) + \VV(\EE(Y|X))$
\end{enumerate}
(Hints: using definition.)

%\item~[20 points]  Let us go back to the coin tossing example. Suppose we toss a coin for $n$ times, \textit{independently}. Each toss we have $\frac{1}{2}$ chance to obtain the head. Let us denote the total number of heads by $c(n)$. Derive the following statistics. You don't need to give the numerical values. You only need to provide the formula.
%\begin{enumerate}
%\item $\EE(c(1))$, $\VV(c(1))$
%\item $\EE(c(10))$, $\VV(c(10))$
%\item $\EE(c(n))$, $\VV(c(n))$
%\end{enumerate} 
%What can you conclude from comparing the expectations and variances with different choices of $n$?  

\item~[15 points] Given a logistic function, $f(\x) = 1/(1+\exp(-\a^\top \x))$ ($\x$ is a vector), derive/calculate the following gradients and Hessian matrices.  
\begin{enumerate}
\item $\nabla f(\x)$
\item $\nabla^2 f(\x)$
\item $\nabla f(\x)$ when $\a = [1,1,1,1,1]^\top$ and $\x = [0,0,0,0,0]^\top$
\item $\nabla^2 f(\x)$  when $\a = [1,1,1,1,1]^\top$ and $\x = [0,0,0,0,0]^\top$
\end{enumerate}
Note that $0 \le f(\x) \le 1$.

\item~[10 points] Show that $g(x) = -\log(f(\x))$ where $f(\x)$ is a logistic function defined as above, is convex. 


\end{enumerate}


\end{document}
%%% Local Variables:
%%% mode: latex
%%% TeX-master: t
%%% End:
